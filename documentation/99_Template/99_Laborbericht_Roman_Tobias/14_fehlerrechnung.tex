Im folgenden Abschnitt geht es darum die Fehlerrechnung zu den oben ausgeführten Messungen durchzuführen (gem. Skript \cite{LOOSER2013}).\\
\\
Es wird unterschieden zwischen dem systematischen Fehler und dem zufälligen
Fehler. Unter \textbf{systematischen Fehlern} bzw. Unsicherheiten werden Fehler
verstanden, welche sich z.B. durch die Versuchsanordnung/ -umgebung einstellen. Ist der systematische Fehler erkannt, lässt er sich meist korrigieren. In diesen Versuchen tritt der systematische Fehler bei den Messungen der Strecken auf.\\
\\
Die für die Fehlerrechnung relevante Formel entspricht der Fitformel \ref{eq:Formel_Lichtgeschwindigkeit_Michelson} und wir hier erneut aufgeführt:\\
\\
%%%%%%%%%%%%%%%%%%%%%%%%%%%%%%%%%%%%%%%%%%%%%%%%%%%%%%%%%%%%%%%%%%%%%%%%%%%%%
\begin{equation}
x = 4\cdot 2 \cdot f \cdot \dfrac{f_{1} \cdot (s_{2}+f_{2})}{c}
\label{eq:Formel_Lichtgeschwindigkeit_Michelson_3}
\end{equation}
%%%%%%%%%%%%%%%%%%%%%%%%%%%%%%%%%%%%%%%%%%%%%%%%%%%%%%%%%%%%%%%%%%%%%%%%%%%%%

Im Kapitel 3 des Skripts \cite{FHNWO2} wird beschrieben, dass die Fehler der Frequenz $f$ und die Verschiebung $x$  bereits in der oben genannten Funktion berücksichtigt sind. Daher müssen einzig die Fehler der Konstanten f1, f2 und s2 berücksichtig werden. Der statistische Fehler wird direkt aus den jeweiligen Fitresultaten entnommen. Die Unsicherheiten der oben genannten Fehlerquellen werden dem Kapitel \ref{sec:Eingestellte Distanzen} entnommen.\\
\\
$S_{f1} = \pm 5 mm$	\\
$S_{f2} = \pm 5 mm$	\\
$S_{s2} = \pm 5 mm$	\\

Die Berechnung des resultierenden Fehlers kann dem Matlabfile im Anhang entnommen werden.\\
\\
Hier folgen die Resultate der Berechnungen:\\
\\
Messung 1: $s_{r1} = 2.6239 \cdot 10^6$\\
Messung 2: $s_{r2} = 4.1187 \cdot 10^6$\\
