Der realisierte Dojo-Prototyp zeigt auf, dass sich die benötigten Komponenten in das von Jana Kalbermatter entworfene Design integrieren lassen.  Alle Sollziele des Pflichtenheftes konnten erfüllt werden. Ausserdem wurde das Wunschziel betreffend der Energieeffizienz umgesetzt. Neben dem Dojo-Prototyp wurden einige BLE-Beacons entwickelt und hergestellt. Eine im Rahmen des Projektes entwickelte PC-Software zur Konfiguration und Auswertung des Dojos vervollständigt das System, mit welchem die Funktion des Dojos demonstriert werden kann. Dazu gehört das Laden des Akkus, das Konfigurieren des Dojos anhand der Ausstellung, das Ein- und Auschecken sowie das Verhalten während des Museumsrundgangs. Nach maximal vier Stunden Ladezeit steht das Gerät für mindestens dreieinhalb Stunden Rundgang zur Verfügung. Audiomaterial von bis zu zwei GB kann auf der Speicherkarte im Dojo abgelegt werden. Der Zugriff erfolgt dabei über USB. Wie im Bericht beschrieben, funktioniert die USB-Verbindung nur unter Linux sowie alten Windowssystemen (Windows 7 und älter).\\ Jedes Dojo kann bei Beginn eines Museumsbesuchs drahtlos auf die Sprache sowie die Zugangsrechte des Benutzers individuell angepasst werden. Befindet man sich im Museum und nähert sich einem Kunstobjekt, wird man durch Vibrieren informiert und kann die zum Objekt gehörende Audiodatei abspielen. Durch Betätigen eines Knopfes lassen sich einzelne Kunstobjekte liken. Beendet der Besucher seinen Besuch, können Informationen über seinen Rundgang exportiert werden. 
Aufgrund des relativ kleinen Dojogehäuses sowie den schnellen Busverbindungen (USB und SDIO) wurde zu Beginn der Projektarbeit das Layouten der Platine als technisch anspruchsvoll eingeschätzt. Dies hat sich jedoch als unproblematisch erwiesen. Die verwendete Hardware liesse sich auch auf eine noch kleinere Platine implementieren. Die technisch anspruchsvollste Tätigkeit war im Nachhinein die Realisierung des BLE-Protokolls zwischen dem Dojo und den verschiedenen Peripheriegeräten.\\
Wie im Bericht bereits erwähnt, wird bei einer allfälligen Weiterentwicklung des Dojos Folgendes empfohlen, um die Einschränkung mit der USB-Verbindung zu beseitigen: Der verwendete Mikrokotroller nRF52832 wird durch seinen Nachfolger den nRF52840 ausgetauscht. Dieser beinhaltet eine USB-fähige Peripherie, womit man auf einen externen USB-Baustein verzichten kann. Weiterhin kann man versuchen, den Zugriff auf die Speicherkarte durch den Mikrokontroller zu erledigen. Diese Hardwareänderungen erlauben es, am verwendeten Softwarekonzept festzuhalten.\\
Im Verlauf des Projektes haben alle Teammitglieder viel gelernt. Besonders interessant war das Anwenden von erlernter Theorie. Mühsam und zeitraubend war der Umgang mit schlecht dokumentierten Komponenten wie dem Soundchip und der USB-SDIO-Brücke. Auf solche Bausteine werden wird in Zukunft falls möglich verzichten. Das realisierte Produkt erfüllt die Erwartungen und mit dem System aus Dojo, Beacons und PC-Software kann eine mögliche Implementation in einem Museum effektiv demonstriert werden. Abschliessend möchten wir noch festhalten, dass wir zufrieden sind. Sowohl mit dem Endprodukt als auch mit der Zusammenarbeit im Team, welche durchs Band gut funktionierte und ohne die wir nicht zu diesem Resultat gekommen wären.

