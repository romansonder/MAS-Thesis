To revolutionize museum visits, people should not only be able to look at museum objects, they should be able to enjoy specific audio information about the art objects in front of them. To achieve these goal, a device called Dojo was designed. The Dojo can recognize Bluetooth beacons placed near an art object and provide specific audio information via a bone conductor about that object. Further, the Dojo can also open doors to additional exhibits. The Device uses a NRF52832 microcontroller with a built in Bluetooth module to scan for nearby Bluetooth beacons. Which can be detected by beacons of 10 meter distance.\\ Every time a new beacon has been localized the Dojo will vibrate and giving a feedback to the user. The audio information itself is stored on a built-in 2 gigabyte micro SD card which provides memory for up to 500 different audio files. An audio chip with an integrated amplifier can read those files and play them through the bone conductor. With the like-button mounted on the case visitors can "like'' art objects to receive a personalised summary at the end of their visit. To select visitor specific preferences such as language or access rights, a user-friendly computer application was developed.\\ The Dojo must only be within range of the computer's transmitter station to transfer the configurations. The same applies to the access control points. The Dojo will automatically exchange its access rights with the transmitter station. To ensure the fast availability of the audio files, the Dojo uses a 335mAh battery which can be recharged in less than 3 hours, allowing the battery to operate for at least 3.5 hours. The Dojo combines the functionality of a universal museum guide and accessibility in one device. It improves the experience of the visitor and simplifies the handling of an exhibit for the museum staff. Thanks to the Dojo, every visit to the museum becomes an unforgetable experience.\\
Keywords: museum guide, bone conductor, digital access control
